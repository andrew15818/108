\documentclass{tufte-handout}
\title{Notes on Pattern Recognition}
\author{Andres Ponce}

\begin{document}
\maketitle
\begin{abstract}
		Pattern Recognition is a subset of Machine Learning and Aritficial 
		Intelligence in general. It's concerned with analyzing data, extracting
		particular features or patterns from that data. We will discuss severl 
		methods for feature extraction, all the way to neural networks and 
		just touch on deep learning!
\end{abstract}

\section{Introduction}
Here, I am using the \texttt{tufte-latex} package for \LaTeX. I thought
the design looked quite nice and wanted to give it a shot! 

\section{Curve Fitting}
	\newthought{There exist} two main types of problems in the field of pattern 
	recognition. First, there is \textit{\textbf{regression}}, and
	there is \textit{\textbf{categorization}}. \footnote{\textbf{regression}
	problems deal with the mapping of an input vector to a continuous space,
	whereas \textbf{categorization} takes an input and places it in a finite 
	and discrete set of different categories.}

	The terms \textit{Artificial Intelligence, Machine Learning,} and 
	\textit{Pattern Recognition} all share common properties. PR $\land$ ML
	$\subset$ AI. ML atempts to make computers take in empirical data and 
	make decisions. AI, more broadly, tries to make computers perform actions
	that were usually thought to be exclusive to humans. 
	
\end{document}
