\documentclass{article}
%useful packages
\usepackage[margin=1in]{geometry}
\usepackage{xeCJK}
\usepackage{helvet}
\usepackage[ruled,vlined]{algorithm2e} %for typesetting algorithms

\setCJKmainfont{Noto Sans Mono CJK TC}
\author{Andres Ponce \\
\and
彭思安}
\title{Introduction to Artificial Intelligence Homework 2 report}
\begin{document}

\maketitle
\section{Background}
Playing games has been an interesting research area in Artificial Intelligence
for several decades, and has led to some of the greatest accomplishments in Artificial
Intelligence with Deep Blue. Several algorithms can be used to model certain types of games 
by modifying a search algorithm. One such an example is the game of Minesweeper. This game can 
be modeled by certain definite rules and thus lends itself very well to being playable by an algorithm.

In this assignment, there was a 6x6 board that consisted of three types of cells: hints, variables, and mines.
Hint nodes contain a numeric value that describes how many bombs there are in its 8 possible surrounding cells.
Given a limited number of bombs, the objective was to find a configuration of the board in which all bombs are used
and every hint node has the amount of adjacent mines described in its value. The objective was to employ
a backtracking search to carry out this search for a valid configuration. 

\section{Solution Description}
The backtracking approach used in this assignment involved a simple main algorithm:

\begin{algorithm}[H]
		\KwResult{valid board configuration}
		\caption{solve\_main(root)}
		\If{
			reject(node)
		}
		{
			return false;
		}
		\If{
			accept(node)
		}{
			return true;
		}
		\For{children in node}
		{
			solve\_main(child);\\
			child = node->next;
		}
\end{algorithm}
Following is a description of those methods.
\subsection{Rejection}
The \texttt{reject} method returns true if the valid board configuration is invalid and should be rejected, and
returns false otherwise. This method takes in a node from the board (i.e. a cell in the board with some 
additional info) and performs a simple check on hint nodes, to see if there are more mine nodes than 
the hint's value allows for. After that, we move on to the accept method.
\subsection{Accepting}
The accpeting method will check whether the current board configuration is a valid one according to the 
constraints given. For a configuration to be valid, we need both all the mines to be used up, and the hint nodes
should all have the exact amount of adjacent nodes described in their values. When this happens, then we know 
we have achieved a valid configuration. Otherwise, this method will return false.
\subsection{Recursion}
Here is where the tree like aspect of the problem unfolds. Since we maintain a data structure with the 
yet unassigned nodes, we get the one from the list to the closest distance. We set a limit to the amount
of children that each node can have (4 in this assignment), and repeat the same process on that node. 
We continue in this way until all nodes have been explored and there are no more available mines to assign.
\end{document}
