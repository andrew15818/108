\documentclass{article}
\usepackage{fontspec}
\usepackage{xeCJK}
\usepackage[margin=1in]{geometry}
\usepackage[default]{sourcesanspro}
%apparently can only use this CJK font?
\setCJKmainfont{Noto Serif CJK TC}

%info
\title{Introduction To Artificial Intelligence: Homework 1 Report}
\author{Andres Ponce(彭思安)\\
0616110}
%%%

\begin{document}
\maketitle
\section{Introduction}
	The first assignment for the course involved a chess board, and placing a knight 
	on that board. Given a starting and ending position, how do we move the knight
	along the ``best" path from the start to the finish? Using several of the methods
	described in the textbook, the task focused on comparing the performance of each one.
	Some criteria may include: completeness, time and space complexity, among other general
	observations.
\subsection{Breadth-First-Search}
	The first algorithm to be implemented was the classic Breadth-First-Search. In general,
	this algorithm keeps a ``frontier" data structure where to-be-expanded nodes are stored.
	Every iteration, we pop a node from this set and examine its children. Since we can move 
	by either $(\pm1 , \pm2)$ or $(\pm2, \pm1)$, every node has a possible 8 children.

	In this implementation, we first check the validity of the specific child node(i.e. if it 
	has already been explored), and if we have not encountered it and is withing the board bounds, 
	we add it to the frontier list, implemented here by \texttt{std::list<Node*>}. If we already explored
	a node,then we already know it does not contain a shortest path, otherwise the algorithm would have
	halted already. 

	To store the individual nodes information, we use a special \texttt{struct Node}, 
	which essentially just stores the node's correspondent x and y values in the board. Then, a quick 
	hash function can turn the x and y values into the index in the node array.

	Breadth-First Search appears to not be the most efficient algorithm to use as is, unless we add
	some form of heuristic when we choose which nodes to explore. Our algorithm will eventually reach 
	the target node, however given a large enough board the time would likely be prohibitive. Since the
	branching factor is 8, since we can make 8 possible moves at every node, the worst-case time complexity
	is $O(d^{8})$, where $d$ is the depth of the solution path.
\subsection{Depth-First-Search}
	For depth-first-search(DFS), we first carry out a thread to the end, and as we are left without valid 
	states to move on to that we start backtracking up the chain. On average, it tends to be slower than
	BFS, since we can explore a large amount of nodes before getting to the target if the target is near
	the starting point. However, the \textit{memory} requirements are quite lower. This is because at any 
	point, only the nodes on the current search path are kept in memory -- the ones we have to search --
	rather than all the nodes at a given level such as BFS. 

	
	From some of the sample tests, DFS resulted in sometimes significantly longer search paths,
	since it would search for paths that lead to dead ends.

	Depending on the problem, DFS would probably not be the most ideal search algorithm. These two first
	algorithms usually have no heuristic to select nodes, so considerable time may be spent on nodes
	that ultimately have no bearing on the final result. Thus, uninformed algorithms in general might
	not be the wisest move. 
	
	An interesting experiment or topic for further discussion given enough time might be 
	to check the actual average difference between random points on the board. 
\subsection{Iterative Deepening Search}
	Iterative Deepening Search, sometimes referred to as \textit{Iterative Deepening Depth-First Search},
	attempts to combine the useful properties of BFS and DFS in one single algorithm. First, we set a limit
	on the depth of the solution, and call a depth-first search on the starting node. Thus, whenever we find
	the solution node, it should be done using the least amount of ndoes possible.

	Every time we don't find the solution node, we increase the limit and try again. This might result
	in the top nodes of the search tree being generated often, however the difference still turns out not
	to be too great, since most nodes reside in the lower levels of the search tree(assuming constant 
	branching factor). Even though the 
	asymptotic running time remains $O(b^{d})$ (same as BFS), the memory complexity is that of DFS $O(bd)$.
	For uninformed strategies with an unknown solution depth, IDS might be the best choice since like DFS
	for a finite space, it is guaranteed to find the solution becuase of the optimality property.

	For our implementation, we use a helper function, since we recursively call the helper function. When first
	entering the helper function, besides from just checking whehter we are at the target, we also have to check
	whether the limit has been reached. Other than that, we proceed in a very similar manner to DFS.
\end{document}
