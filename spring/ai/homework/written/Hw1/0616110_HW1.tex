\documentclass{article}
\usepackage{xeCJK} %for Chinese name
%\usepackage{helvet}
\usepackage[margin=1in]{geometry}

\renewcommand{\familydefault}{\sfdefault}
\title{Introduction To Artificial Intelligence Written Assignment 1}
\author{Andr\'es Ponce(彭思安) \\
\and
0616110}
\begin{document}
\maketitle
\section{For the two videos shown in our first class(one on Robot Mouse Races, and the other one Robothespian)
, describe their respective PEAS.}
PEAS is an acronym for \textbf{P}erformance Measure, \textbf{E}nvironment, \textbf{A}ctuators, and
\textbf{S}ensors. These are 5 elements of a rational agent. A \textbf{performance measure} 
refers to the question:
How do we measure whether an agent is performing well? This will determine what factors need to be changed
for the agent to accomplish its objective. The \textbf{environment} refers to the external world in which
our object interacts(a game board, a road, etc...). The \textbf{Actuators} are the possible actions that 
our agent can perform, such as a move in a game, or turning right or left on an intersection. The sensors
are the parts of the agent that detect the external conditions and provide the information the agent uses
to make its decision.

RoboThespian is a humanoid-looking robot designed to complete multiple tasks that have some resemblance to 
human tasks. It is advertised as being able to talk and interact with other humans, and even simulate 
emotional responses. For RoboThespian, PEAS might look something like:
\begin{itemize}
	\item \textbf{Performance Measure} Since RoboThespian was designed originally to interact with others,	
			a possible perfomance measure could be a positive reaction from the people it interacts with,
			such as a laugh. 
			However, since RoboThespian can perform multiple activites (sing, act, etc...), an approach with 
			multiple performance measures for each activity might be more accurate. The first approach might	
			be more suitable if RoboThespian is interacting casually with other people rather than doing a
			specific activity.
	\item \textbf{Environment} Since RoboThespian relies on interaction, the environment would have to be 
			a place with one or more people for it to perform some actions and receive input in acoordance 
			to the rest of the model.
	\item \textbf{Actuators} Regardless of what acitivity RoboThespian is performing, some possible 
			actions for it to take might be speaking (possibly with different tone of voice and pitch,etc...),
			singing, moving its limbs, blinking, or some other physical response.
	\item \textbf{Sensors} RoboThespian could receive its input from the audience via a camera, physical 
			sensors on it's hands, or some other way to gauge the audience's response.
\end{itemize}

Robot Mouse Races, on the other hand, describe how a ``mouse", programmed in a certain way, can run through 
a maze without help and even find the best route through it.  Some of this agent's features are described below:

\begin{itemize}
	\item \textbf{Performance Measure} The performance measure could be quantified in how long the mouse takes to 
			complete the maze. 
	\item \textbf{Environment} The environment for the mouse would be a maze, the space where it receives inputs
			and makes decisions.
	\item \textbf{Actuators} The mouse would could either move forward, backward, turn left, turn right,
			or stop. 
	\item \textbf{Sensors} The sensors on the mouse would probably include some sort of camera to see what lies
			in front or to the sides. Also it might include some proximity sensor to indicate when it will
			hit a wall or when there is an open turn it can make, possibly at an intersection.
\end{itemize}

\section{In a popular English word game, the goal is to convert a given English word to another given English
word of the same length by changing one letter at a tie. Each intermediate word needs to be in a standard 
dictionary. For example, if the initial word is DOG and the destination word is CAT, the following is a possible
series of words: $DOG\rightarrow DOT\rightarrow HOT\rightarrow HAT \rightarrow CAT$. 
Now you are given this set of two-letter English words as your 
dictionary: \{AN, AM, AS, AT, AX, BE, BY, GO, HE, HI, IT, IS, IN, IF, ME, MY, NO, OF, OH, OK, ON, OR,
OX, SO, TO, UP, US, WE\}}
	\subsection{Find a solution from AT to IN using breadth-first search(BFS). When expanding a node,
		generate its children in alphabetical order. Use no repeated state checking. Give separated lists of 
		all the generated nodes and expanded nodes, both in the correct order.}
		\label{subsec:bfs}
		Since we can only change oen letter at a time, at each node we can have up to two children, 
		one where we change the first letter and one where we change the second letter in the word.	From the 
		starting node AT, two possible solutions might look like: 
		\[ \textrm{AT} \rightarrow \textrm{AN} \rightarrow \textrm{IN}\]
		\[ \textrm{AT} \rightarrow \textrm{IT} \rightarrow \textrm{IN}\]

		The nodes expanded would be: AT, AN. 

		All the nodes generated would be: AT, AN, IT, IN.
	\subsection{Explain why Hamming distance (the number of positions where the two words have 
		different letters) can be used as an admissible heuristic for this problem. You need to 
		provide a reasonable explanation.}
		\label{subsec:hamming}
		An \textbf{admissible} heuristic is one where the function $h(n)$ never overestimates the
		the true cost from the current node $n$ to the goal node. 
		
		In this problem, $h(n) \in \{0,1,2\}$, since for two letter words, the most they can differ is in their
		two letters, and will differ by none if they are the same word. We have to then show that $h(n)$ will 
		not overestimate whenever it is 0,1, or 2. 

		If $h(n) = 0$, then both of the letters in $n$ are the same as the ones in the target, which would 
		imply we reached the target and thus need to change 0 letters.
	
		If $h(n = 1)$, then the current word and the target differ by only one. In this case, the minimum amount
		of changes needed to reach the target word is 1, so $h$ will does not overestimate.

		Similar to the case where $h(n) = 1$, if $h(n = 2)$ then there are at least two words in between
		$n$ and our target, where one letter is different in each. 

		Since the true cost of the function will always be at least the Hamming distance, $h(n)$ is an 
		admissible heuristic.
	\subsection{Repeat (\ref{subsec:bfs}) using A* search with the heuristic in (\ref{subsec:hamming}).}
\end{document}

