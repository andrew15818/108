\documentclass{tufte-handout}
\usepackage{graphicx}
%\usepackage{equation}

\title{Final Notes for Intro to Artificial Intelligence Final Exam}
\author{Andr\'es Ponce}

\begin{document}
\maketitle
\begin{abstract}
Topics covered in this final exam include search algorithms, constraint satisfaction, 
and logical satisfiability.
\end{abstract}

\section{Search Algorithms}
\subsection{Breadth-First Search}
With this algorithm, remember we have to search for the \textit{shallowest unexpanded node}.
This means that we will check the tree/graph layer by layer before moving on to the next 
	yet unexpanded node.
The data structure used for this algorithm will probably just be a FIFO structure, since we 
	first want to examine the nodes that are shallower than the current one, so we should pop
	the node that has been the longest in the structure.

\begin{enumerate}
	\item \textbf{Completeness}: Yes, on finite graphs. 
	\item \textbf{Time Complexity}: $O(b^{d})$
	\item \textbf{Space Complexity}: $O(bd)$
	\item \textbf{Optimality}: Yes, for unit costs. Why? If there is a unit cost, then the first time
			we encounter the node, its value will be guaranteed to have the lowest value. If it did not
			have the lowest value, we would have encountered it previously, right?
\end{enumerate}
\subsection{Depth-First Search}
Depth-First Search works quite differently than Breadth-First search. With DFS, we expand the deepest,
	unexpanded node. 
What this means is that we traverse the tree/graph in a single path, following the children of the current
	child until we can do so no more.
Once this happens, we then backtrack up to the deepest unexpanded node, possibly along the same general
	path that we were travelling before.

\begin{enumerate}
	\item \textbf{Completeness}: Nope. Might repeat given loops in the state space.
	\item \textbf{Time Complexity}: $O(b^{m})$, because we could potentially reach the maximum state space
		distance before finding the target node.
	\item \textbf{Space Complexity}: $O(bm)$, since we have to sotre all nodes in memory
	\item \textbf{Optimality}:Nope, it may search other paths, which might cause it to terminate before
		actually reaching the best solution.
\end{enumerate}
It is complete for finite state spaces, probably becuase it searches all the nodes eventually, thus finding
	a solution if there is one.

\subsection{Depth-Limited Search}
This algorithm is essentially DFS but having a maximum depth be $l$. What does this avoid? 
The so-called loops which may result are avoided, however we would need to be careful to not limit
	ourselves to not finding the wrong answer if we set a small $l$.

\subsection{Iterative Deepening Search}
Here, we consistently run depth-limited search on several of the nodes.
We increase the maximum depth $l$ every time we run the algorithm.

\begin{enumerate}
	\item \textbf{Completeness}: Yes, for finite $d$.
	\item \textbf{Time complexity}: $O(b^{d})$
		\footnote{Remember $d$ is the length of the maximum length of the solution.} 
	\item \textbf{Space Complexity}: $O(bd)$
	\item \textbf{Optimality}: Yes.
\end{enumerate}

\subsection{Uniform-Cost Search}
With this algorithm, we expand the least-cost unexpanded node. 
This means that as we get closer to the target node, we might also see a reduction in the value of the 
	node.
\begin{enumerate}
	\item \textbf{Completeness}: Yes, if cost $\geq\epsilon$
	\item \textbf{Time Complexity}:
	\item \textbf{Space Complexity}
	\item \textbf{Optimality}: Yes
\end{enumerate}
\subsection{Informed Search Strategies}
If we remember, \textit{un}-informed search strategies would use the 
With informed search, we can use more information at each stage other than just the parent and children,
	for example, we can have an idea of how much we still have to go until the target node, for example.

With this new ability, we have a couple more tools to use:
\begin{enumerate}
	\item \textbf{Best-First Search}: Now, we can expand the node that looks to lead us to the 
		best solution (e.g. the node whose heuristic value is the lowest).
	\item \textbf{Evaluation Function}: We now can have a function $f(n)$ which tells us which node
		to choose for expansion. (Useful in A*, for example).
	\item \textbf{Heuristic Function}: We can now have an educated guess of how much we have to reach the 
		
\end{enumerate}

For example, the first algorithm we can study is a simple \textbf{Greedy Best First} algorithm. 
Using this algorithm, we merely select the node that has the best heuristic value out of the bunch.
However, this method is neither complete nor optimal, why?
\footnote{I guess there might be a scenario where at one stage $n$, we choose the one with the lowest
	heuristic value, but at a lager stage $n+j$ it turns out the combined sum was lower for a second path.}

\subsection{A* Search}
Here, the algorithm relies on one formula
\begin{equation}
	f(n) = g(n) + h(n)
\end{equation}
The $f(n)$ value, as mentioned, represents the total combined value for the node.
This value is the sum of $g(n)$, i.e. the cost of moving from the starting node to the current node,
	and $h(n)$, which is the cost of going from the current node's children to the target node.
\begin{enumerate}
	\item \textbf{Completeness}: Yes, unless there are infinitely many nodes with cost $f(n)<C$, where C is the 
		optimal cost to goal state.
	\item \textbf{Time Complexity}: Exponential
	\item \textbf{Space Complexity}: Keeps all the nodes in memory, so not pretty efficient
	\item \textbf{Optimality}: Yes, given some conditions on the heuristic:
\end{enumerate}
For A* to really live up to its potential, we need an \textbf{admissible} heuristic. 
An admissible heuristic is one which does not overestimate the true cost of the node to the target.
\footnote{Ohterwise, if it overestimated sometimes, we could choose a less efficient route by accident?}

Another condition for A* optimality is a \textbf{consistent} heuristic. A consistent heuristic is one which
satisfies the \textbf{triangle inequality}:
\begin{equation}
	 h(n) \leq c(n,a,n') + h(n)'
\end{equation}
In this equation, $c(n,a,n')$ is the cost of going from the current node $n$ 
	to the next one in the path $n'$ by using action $a$.
\footnote{Consistent Heuristics are always admissible, not so the other way around.}

\subsection{Iterative Deepening A* Search}
We can also apply an increasing limit to A* in order to reduce the amount of nodes stored in memory at once.

\section{Constraint Satisfaction Problems(CSP)}
We can model problems as those involving some sort of logical inference. 
The necessary logical rules can also be modeled in the  program. 

An important concept in logic is \textbf{entailment}. 
Entailment refers to when we have statements $a$ and $b$, whenever $a$ is true, then $b$ is true.
This is written as $a\models b$.

We then have \textbf{inference}, where we can draw new conclusions from logical statements that we 
	already know.
This is written as $KB\vdash \alpha$.
We have two types of inference algorithms: \textbf{sound} and \textbf{complete} algorithms.
Sound inference algorithms are ones that
\footnote{Soundess here means that if there we can get $\alpha$ from our knowledge base with an algorithm,
	then $\alpha$ is always true whenever the KB is true.}
\[ KB\vdash\alpha\Rightarrow KB\models\alpha\]
Then, we have a complete inference algorithm:
\footnote{Here, we mean that if $\alpha$ is true whenever our KB is true, then there exists an algorithm
	to derive it from the knowledge base.}

\[KB\models\alpha\Rightarrow KB\vdash_{i}\alpha\]
\subsection{Propositional Logic}
In this type of logic we just have simple sentences, and we evaluate the truth, validity and soundness of
	the clauses based on the \textbf{truth value} of the statements.
Here are some useful terms in propositional logic.
\begin{itemize}
		\item \textbf{logical equivalence}: Two statements that are equivalent in all the same 
				set of models.
		\item \textbf{valid}: A sentence that is valid in \textit{all} models.
		\item \textbf{satisfiable}: A sentence that is true in \textit{some} models.
\end{itemize}

\textbf{Conjunctive Normal Form} refers to a set of clauses that are joined together with an and.

\subsection{Resolution Algorithm}
To show that  $KB \land \alpha$ is true, we have to show that $KB \land \neg \alpha$ is unsatisfiable.

\subsection{First Order Logic}
With First Order Logic, we have statements that use variables and can be represented by a subset of the 
	domain.
We have \textbf{functions} that apply a relation between the \textbf{domain} the environment. 

Two important tools in first-order logic are \textbf{universal} and \textbf{existential} quantification.
The former applies to every possible object in the model.
\[ \forall AtNCTU(x) \Rightarrow Smart(x)\]
says that every student at NCTU is smart.
The same example using existential quantification is
\[ \exists AtNCTU(x) ^ Smart(x)\]
which says that there is an x such that x is a student at NCTU and x is smart.
\end{document}
