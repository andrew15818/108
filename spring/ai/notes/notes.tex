\documentclass{tufte-handout}
\usepackage{graphicx}
\usepackage[ruled,vlined]{algorithm2e}
\graphicspath{ {./images/} }


\title{Introduction to Artificial Intelligence}
\author{Andres Ponce}

\begin{document}
\maketitle
\begin{abstract}
	Suppose we have a game with two agents. The two agents each
	make moves that are optimal for them, according to some constraints. 
	They may not know all the environment, but they can still make 
	optimal decisions from what they do know.
\end{abstract}
\section{Chapter 5: Adversarial Search}

\subsection{5.2 Optimal Decisions in Games}
	Because adversarial circumstances allow the agents to influence each 
	other's decisions, we need some way to still arrive at an optimal 
	solution for the agent we wish to "win". To recap, we have a 
	\textbf{Min-Max} situation.\footnote{In this situation, the \textbf{max}
	agent takes the option with highest value, and the min agent will choose
	the one with lowest value as a counter move.} 

	Remember the \smallcaps{Utility} function tells us how useful a certain 
	end state is for our given configuration. The utility is defined by adding
	the utility of the leaf node and the sequence of min-max moves from the node
	to the leaf node. 

	The \smallcaps{minmax} algorithm backs up from a leaf node, adding
	either the max or the min utility values as it backs up, depending whether
	it is \textbf{Max}'s turn or \textbf{Min}'s. This assumes that both agents 
	always play optimally and know what the best move is at every state. 
	
	Although this algorithm will search every state in DFS fashion, it will 
	clearly be too slow for anything resembling a real game. However, other algos 
	use this as their starting point. 
	
	When we have \textbf{multiple agents} in a single game, we need a vector of 
	utility values for each state and every agent. Two weaker agents could form 
	\textbf{alliances} to bring down the bigger agent, however, since this is purely 
	selfish behavior, the alliance will probably dissovle when there is no more threat 
	from that agent. 
	
\subsection{5.3 Alpha-Beta Pruning}
	The idea behind \textbf{pruning} is that we can remove one of the large exponents 
	behind the search. The way we remove the steps is by realizing that for a node 
	$n$ somewhere in the tree, if there is a better move anywhere above it, then $n$
	will never be reeached by the minmax algorithm. This means that we can safely
	not even search it. Since we continue to switch between Max and Min nodes, if the 
	node to be chosen by a Max node is greater than the node to be chosen by a Min node,
	we node that the node chosen by a Min node will never be chosen by a Max node,
	so we can safely discard it.\footnote{The \textbf{alpha} and \textbf{beta} values refer 
	to the maximum value of the Max and Min players, respectively. If $\alpha$ > $\beta$,
	then we can say we will never choose $\beta$ as our best move.} 

	When we are dealing with search trees, we have to be careful to avoid 
	\textbf{transposition}, since that can really throw off our attempt.
	\footnote{Transposition refers to multiple move sequences that lead to the same end
	result.}
\subsection{5.4 Imperfect Real-Time-Decisions}
	Supposing we are playing chess again(while evbdy. else plays checkers). Since the search
	space is too large to make a move quickly while searching all the different possibilities,
	we could cut off the search after a certain number of nodes. This would allow us to 
	choose a move in a certain amount of time since we are making on judgement on 
	the remainder of the \smallcaps{eval}  function.	

	
\end{document}
