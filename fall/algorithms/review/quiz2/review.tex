\documentclass[landscape,4pt,a4paper]{article}
\usepackage{multicol}
\usepackage[margin=0.3in]{geometry}
%\usepackage{algorithmicx}
%\usepackage[noend]{algpseudocode}
\usepackage{amsmath}

\begin{document}
	\raggedright
	\begin{multicols*}{5}
		\section{Graph-Traversal}
			\subsection{Breadth-First Search}
				Given a graph $G=(V,E)$, we want to discover all nodes a distance $dis(s,v)$.
				For BFS, we maintain a set $S$ of all the nodes that we have travelled to. 
				We push all of the vertices in $\text{Adj}[i]$. Also, we maintain a the parent of 
				every node we find the shortest path of. Since we only place each node once in the set $S$,
				in total our algorithm will take $O(n+m)$.
			\subsection{Max trees}
				If we run BFS algo twice from an arbitrary node, the greatest $\text{dis}[d]$ will be 
				one of the points on the pair $(o_{1},o_{2})$. Why? If there exists a longer 
				path than $(o_{1},o_{i2})$, then there has to be some contradiction or cycle, which is not 
				allowed by the definition of a tree.
			\subsection{DFS-Visit}
				In DFS, we move through the most recent currently unexplored node. As soon as we find a node
				that is not yet explored, we mark it as ``discovered", and immediately move to explore its
				edges. When we finish exploring every node, we mark it as ``finished". However, DFS-visit 
				only acts on the nodes reachable from $s$.
			\subsection{DFS}
				If we want to act on all the nodes of the graph, we need to consider the nodes not reachable 
				from $s$. For two nodes $(u,v)$, if there is overlap between $[u.d,v.d]\text{and}[u.f,v.f]$,
				then one is the ancestor of the other. Why is DFS said to create a \textit{forest}? Since 
				not all the elements might be reached with just the initial call of DFS, we might need to 
				call it on other nodes originally unreachable from our starting node.
			\subsection{Edges}
				Four types w.r.t. DFS-forest F
				\begin{itemize}
					\item{\textbf{tree edges}}: edges $(u,v)\in$ F.
					\item{\textbf{back edges}}: edges $(u,v)$ connecting $u$ to an ancestor.
					\item{\textbf{forward edges}}: edges $(u,v)$ connecting node to descendant.
					\item{\textbf{cross edges}}: all other edges(b/w disjoint trees....)
				\end{itemize}
			\subsection{Topological Sort}
				We try to find an ordering of ndoes in a DAG. We can do this by comparing the finishing 
				times of the DFS on each node.
	\end{multicols*}
\end{document}
