\documentclass{article}
\usepackage[margin=0.5in]{geometry}
\usepackage[backend=biber,style=chicago-authordate]{biblatex}

\addbibresource{reference.bib}
\title{Bioshock: The economics of revolution}
\author{Andres Ponce \\
\and
0616110
}
\begin{document}
\maketitle
``No Gods, or Kings, Only Man"\textcite{Bioshock} mark the entry to Rapture, the underwater city in the middle 
of the Atlantic Ocean
that serves as the setting for the 2007 game Bioshock. 
The product of respected video game designer Ken Levine, Bioshock earned much praise upon
its release for its treatment of the ideal society, one in which man investigates and produces without any 
constraints. Rapture's creator, Andrew Ryan, sought to create a society untethered by political or social 
norms; where all that a man produced belonged to him, and to him alone. The main character Jack begins the 
game on a flight crossing the Atlantic, which mysteriously crash lands, leaving him as the only survivor.
When a gigantic lighthouse presents the only escape out of a cold ocean, he unquestioningly makes his way
to the ornate entryway to Rapture proper. As he takes a personal submarine, called a ``bathysphere" down 
ever deeper into the ocean, he is greeted by a pre-recorded message of Andrew Ryan, which serves as the 
introduction to the ideas on which Rapture were built. In a classic mid-century voice, he says ``...Is a man
not entitled to the sweat of his brow? No, says the man in Washington. It belongs to the poor. No, says the 
man in the Vatican. It belongs to God. No, says the man in Moscow. It belongs to everyone. I rejected those
answers."\textcite{Bioshock}

The city goes to great lengths to emphasize the elevation of man almost as a god; the art deco design of the
city features sculptures of men in heroic positions. However, as soon as the player enters the starting 
building, a very different image of the city becomes clear. The person who opened the gate for our bathysphere
is murdered by some humanoid figure, which we learn are called ``splicers". For the remainder of the game, 
the player faces increasingly dangerous splicers, which we discover are humans that have been corrupted by 
a drug calles ADAM.

The most striking part of the game's story for many turns out to be the ideas that drove Ryan to build Rapture
in the first place. According to the Levine, "It's ironic because Ryan doesn't believe in God. He believes 
in man as a god. So when he built this city, all the best people got taken away to this place. The notion of 
The Rapture is that all the believers will be taken up to this ideal place. Well, this is his Rapture; this is
where all his ideal people got brought to and were spirited away from the rest of the world."
\textcite{LevineInterview} The name Rapture then meant a place where the brightest minds were able to 
be productive unbothered by ``petty morality" as Ryan  puts it in the introductory speech.  The question 
which first arises seems to be: given that people were unbound to political regulations, morality, religion,
or ethics, could a society such as Rapture prosper? The game certainly aims to show that it does not seem 
possible, and this essay shall attempt to show why that should be the case. 

\section{What led to Rapture?}
Before talking about the eventual decline of Rapture, what drove Ryan to build it at all? Andrew Ryan's name
itself is an anagram for the answer: the philosophy of 20th century Russian-American writer Ayn Rand. Rand 
developed her own unique philosophy called Objectivism, which in her own words "...is the concept of man
as a heroic being, with his own happiness as the moral purpose of his life, with productive achievement as
his noblest activity, and reason as his only moral absolute."\textcite{AtlasShrugged}

Of course, Ryan's embrace of this philosophy stems from his dissatisfaction with capitalism and communism,
which at the time of the story (May 1960), fought an ideological battle on the surface. Ryan's 
main issue with these two systems seems to stem from the idea of what a man produces being taken away. For 
him, wealth and ``the work of man's brow" being redistributed to those who have not worked for it strips man
away of his motivation to keep improving. In an early level of the game, once the player learns to ``hack"
vending machines and buys health and ammo upgrades for cheaper, he comes on the audio system and says ``...
I should not need to remind each and every citizen of Rapture that free enterprise is the foundation on which
our society has been established. Parasites will be punished."\textcite{Bioshock} He routinely refers to 
people on the surface,without distinction to nationality or preferred system, implying that people 
of all economic systems commit the same mistakes.

Capitalism, in its broadest sense, maintains that individuals own the means of production. Through the use of
wage labor, people receive payment for the work they do for the capitalist class. \textcite{ComparingEconomics}
However, some people argue that it remains in the government's responsibility to care after the lower classes in
society.  This latter claim would seem to match what Ryan dislikes about the system; the need for some to be 
taken care of by a big government using the wealth created by those better off. Communism, on the other hand,
calls for the abolition of private property in general. Engels says in \textit{The Principles of Communism}:
``Private property must, therefore, be abolished and in its place must come the common utilization of all 
instruments of production and the distribution of all products according to common agreement...communal
ownership of all goods"\textcite{PrinciplesOfCommunism}

Ryan, maintaining that the individual has a right to his or her own labor, as well as the wealth or gains that 
come from it, unsurprisingly has a strong dislike for both systems; after all, neither will allow him to 
be the true owner of what he has produced. However, once Ryan had built Rapture to fulfill his dream of 
true free enterprise, things started to go awry. 
\printbibliography
\end{document}
