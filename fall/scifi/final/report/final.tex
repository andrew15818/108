\documentclass{article}
\usepackage[margin=0.5in]{geometry}
\usepackage[backend=biber,style=chicago-authordate]{biblatex}

\addbibresource{reference.bib}
\title{Bioshock: The economics of revolution}
\author{Andres Ponce \\
\and
0616110
}
\begin{document}
\maketitle
``No Gods, or Kings, Only Man"\textcite{Bioshock} mark the entry to Rapture, the underwater city in the middle 
of the Atlantic Ocean
that serves as the setting for the 2007 game Bioshock. 
The product of respected video game designer Ken Levine, Bioshock earned much praise upon
its release for its treatment of the ideal society, one in which man investigates and produces without any 
constraints. Rapture's creator, Andrew Ryan, sought to create a society untethered by political or social 
norms; where all that a man produced belonged to him, and to him alone. The main character Jack begins the 
game on a flight crossing the Atlantic, which mysteriously crash lands, leaving him as the only survivor.
When a gigantic lighthouse presents the only escape out of a cold ocean, he unquestioningly makes his way
to the ornate entryway to Rapture proper. As he takes a personal submarine, called a ``bathysphere" down 
ever deeper into the ocean, he listens to a pre-recorded message of Andrew Ryan, which serves as the 
introduction to the ideas on which Rapture were built. In a classic mid-century voice, he says ``...Is a man
not entitled to the sweat of his brow? No, says the man in Washington. It belongs to the poor. No, says the 
man in the Vatican. It belongs to God. No, says the man in Moscow. It belongs to everyone. I rejected those
answers."\textcite{Bioshock}

The city goes to great lengths to emphasize the elevation of man almost as a god; the art deco design of the
city features sculptures of men in heroic positions. However, as soon as the player enters the starting 
building, a very different image of the city becomes clear. The person who opened the gate for our bathysphere
gets murdered by some humanoid fugure, which we learn are called ``splicers". For the remainder of the game, 
the player faces increasingly dangerous splicers, which we discover are humans that have been corrupted by 
a drug calles ADAM.

The most striking part of the game's story for many turns out to be the ideas that drove Ryan to build Rapture
in the first place. According to the Levine, "It's ironic because Ryan doesn't believe in God. He believes 
in man as a god. So when he built this city, all the best people got taken away to this place. The notion of 
The Rapture is that all the believers will be taken up to this ideal place. Well, this is his Rapture; this is
where all his ideal people got brought to and were spirited away from the rest of the world."
\textcite{LevineInterview} The name Rapture then meant a place where the brightest minds were able to 
be productive unbothered by ``petty morality" as Ryan  puts it in the introductory speech.  The question 
which first arises seems to be: given that people were unbound to political regulations, morality, religion,
or ethics, could a society such as Rapture prosper? The game certainly aims to show that it does not seem 
possible, and this essay shall attempt to show why that should be the case. 

\section{What led to Rapture?}
Before talking about the eventual decline of Rapture, what drove Ryan to build it at all? Andrew Ryan's name
itself stands for the answer with an anagram : the philosophy of 
20th century Russian-American writer Ayn Rand. Rand 
developed her own unique philosophy called Objectivism, which in her own words "...is the concept of man
as a heroic being, with his own happiness as the moral purpose of his life, with productive achievement as
his noblest activity, and reason as his only moral absolute."\textcite{AtlasShrugged}

Of course, Ryan's embrace of this philosophy stems from his dissatisfaction with capitalism and communism,
which at the time of the story (May 1960), fought an ideological battle on the surface. Ryan's 
main issue with these two systems seems to stem from the idea of what a man produces being taken away. For 
him, wealth and ``the work of man's brow" being redistributed to those who have not worked for it strips man
away of his motivation to keep improving. In an early level of the game, once the player learns to ``hack"
vending machines and buys health and ammo upgrades for cheaper, he comes on the audio system and says ``...
I should not need to remind each and every citizen of Rapture that free enterprise is the foundation on which
our society has been established. Parasites will be punished."\textcite{Bioshock} He routinely refers to 
people on the surface,without distinction to nationality or preferred system, implying that people 
of all economic systems commit the same mistakes.

Capitalism, in its broadest sense, maintains that individuals own the means of production. Through the use of
wage labor, people receive payment for the work they do for the capitalist class. \textcite{ComparingEconomics}
However, some people argue that it remains in the government's responsibility to care after the lower classes in
society.  This latter claim would seem to match what Ryan dislikes about the system; the need for some to be 
taken care of by a big government using the wealth created by those better off. Communism, on the other hand,
calls for the abolition of private property in general. Engels says in \textit{The Principles of Communism}:
``Private property must, therefore, be abolished and in its place must come the common utilization of all 
instruments of production and the distribution of all products according to common agreement...communal
ownership of all goods"\textcite{PrinciplesOfCommunism}

Ryan, maintaining that the individual has a right to his or her own labor, as well as the wealth or gains that 
come from it, unsurprisingly has a strong dislike for both systems; after all, neither will allow him to 
be the true owner of what he has produced. However, once Ryan had built Rapture to fulfill his dream of 
true free enterprise, things started to go awry. 

\section{The Rise and Fall of the Bioshock Empire}
At a first glance, Rapture seems to represent the ideal city on a hill. It managed to draw like-minded men and
women of science, art, and industry. However, the slightest glance beyond the superficial reveal that Rapture
has utterly failed in remvoing itself from the problems of all societies. These include: crafty inhabitants
who bend the rules for personal profit, drugs, and a deeper problem concerning the base of their morals. 

Although Ryan founded Rapture in order to have as minimal amount of government interference in daily, as well 
as business life, he places retrictions that would have disastrous ripple effects.  Although man retains
the right to enjoy the fruits of his or her labor, he does not enjoy the ability to communicate with the 
surface. This poses the obvious problem of stifling innovation and trade, since the market ends where the 
surrounding ocean begins.The game's antagonist, Fontaine, takes advantage of this flaw and quickly builds 
illegal trade routes to the
surface, through which he manages to bring in drugs, among other illegal items. About midway through the game,
the player encounters  Andrew Ryan himself, who reveals that the player had been genetically conditioned to
mindlessly follow the orders once he heard the phrase ``Would you kindly...". This plan, orchestrated by 
Fontaine himself, does more to show the pettiness of human beings and what lengths they go to for power than 
any other aspect of the game. Power struggles such as these are commonplace in most organizations; 
gethering the greatest minds certainly did not end the basest struggles that arise among human beings. 

The second reason why Rapture finds itself in a state of stagnation and deterioration also cuases the player
to have superhuman powers: drugs, and more specifically Adam. Right as the player enters the city, he 
encounters a strange substance, which his aid Atlas tells him changes his genetic code. As drugs do, Adam 
drove the citizens of Rapture to obsession, and a disturbing industry arose out of its production and 
distribution. Here Ryan's ideologies also prolong the situtaion. Due to his aversion to legislation, he relies
solely on his ``Great Chain" of industry to itself resolve this problem. This drug, which has the property to 
accelerate healing and provides the player with telekinetic, electrical, and incendiary power, has a dark side 
to its  production. Adam results from the secretions of a special type of sea slug. In order to mass produce
Adam, producers take little girls from their mothers, and genetically modify them to act as hosts to the slugs. 
In order to protect these little girls, or ``Little Sisters", other genetically modified humans, called
``Big Daddies" protect them due to psychological manipulation. Although even Ryan recognizes this practice,
and calls Little Sisters ``abominations" in a radio transmission to the player. Ryan's inability to act even 
when faced with an evil enterprise, points to an even deeper problem regarding the basis of Rapture itself. 

These two examples might lead the audience to ask the question: What morality underlies Rapture? Traditional 
morality intentionally played very little part in Rapture's founding, with Ryan specifically mentioning that
as a reason to escape the surface. In Rapture, the individual clearly does not enjoy as high a position as 
the free market, else Ryan would have stepped in to eliminate the Adam business, rather than relying on the 
``Great Chain" of society to slowly but surely move society in the right direction. Although Ryan and 
Fontaine take center stage for the majority of the game, other major inhabitants do not fall far behind 
in terms of morality, or lack thereof. Dr. Steinman from the beginning of the game performed unregulated
plastic surgery on patients who wished for unnatural beauty; Sander Cohen, a deranged artist who takes pleasure
in murdering his victims an artistic manner, has control over an entire section of Rapture; 
Dr. Brigid Tenenbaum, a former contentration
camp survivor, takes charge of manufacturing the Little Sisters. Given this context, it becomes clear that 
outright doing away with any concept of right and wrong sets society on an ever steeper slope downward. If 
the economy takes a greater value over the inhabitants of the city, why should Ryan bother to care about little 
girls being mutated? Why should the ongoing genetic experiments be of any concern to the authorities? Herein
lies the problem; Rapture was founded with the idea of doing the exact opposite of what surface societies did:
increasing the size and authority of government. However, the opposite of that allowed evil to flourish, and do
so while following society's ideals quite closely. This lack of grounding remains the main reason why Rapture
could never hope to achieve something that was lasting: if there does not exist a solid moral grounding,
anything can and eventually become permissible. As Levine also put it, "But they [science] also get used for 
incredible ends. It's always a yin and a yang; and the question is: do you take it tot he extreme or do you take 
a step back and go 'wait a minute'?"\textcite{LevineInterview}. 

\section{Conclusion}
Bioshock, as a video game, pushed the traditional image of games as mindless, cheap fun. Its unique setting,
sound design, and narrative were among the first to take a philosophical position such as Objectivism and 
critique it in such a form. The sequel, \textit{Bioshock Infinite} continues the same trend of taking 
an ideology and taking it to its logical extreme, in that case it deals with the idea of industrialization and
American Exceptionalism. 

Rapture, being founded on ideas intentionally contrary to those on the surface, finds itself
with much of the same problems. Everybody who suppsedly propels Rapture forward ends up harming Rapture in
the long term. This stems from the lack of any sort of concrete ideals that ground Rapture with a common goal. 
Rapture in the end has no way of achieving the same lasting influence as some of the surface societies due to 
the fact that when some new discovery actually causes more harm than good, there does not exist a compelling
ideological reason to put an end to it, since production and the market take a higher position than human 
dignity. 


\printbibliography 
\end{document}
